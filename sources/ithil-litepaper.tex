
\RequirePackage{snapshot}
\documentclass [10pt, fancyhdr, twoside] {article}
\usepackage{float, graphicx, caption, amssymb, natbib, amsmath}
\usepackage[usenames,dvipsnames]{color}
\usepackage{tabulary}
\usepackage [left=2.5cm, top=2.5cm, bottom=2.5cm, right=3cm] {geometry}  %% see geometry.pdf on how to lay out the page. There's lots.
\geometry{a4paper} %% or letter or a5paper or ... etc
\usepackage{fancyhdr}
\usepackage{xcolor}
\usepackage[scaled]{helvet}
\renewcommand*\familydefault{\sfdefault} %% Only if the base font of the document is to be sans serif

\usepackage[left]{lineno}
\usepackage[ddmmyyyy,hhmmss]{datetime}

\renewcommand{\linenumberfont}{\normalfont\tiny\color{gray}}


\pagestyle{fancy}

\fancyhead{}
\fancyfoot{}

\fancyhead[RO,LE]{Ithil Finance - Whitepaper}

\usepackage{blindtext}

\newcounter {note}
\stepcounter{note}

\renewcommand{\abstractname}{Abstract}

\newcommand {\Note} [1] {
    \marginpar {
        \tiny {
            {\color{gray}{\thenote  \  #1}}
            }
        }
    \stepcounter {note}
}

\newcommand {\MNote} [1] {
    \marginpar {
        \tiny {
            {\color{gray}{#1 }}
            }
        }
}

\begin{document}

\title{Ithil Finance}
\author {G. Casanova,  S. M. Chiarello, M. Ugolotti}

\date{\today}

\maketitle

\begin{abstract}
Ithil is a margin trading and staking protocol, where liquidity providers can earn passive income generated by the margin trading activity. Liquidity is protected thanks to a decentralized liquidation system, for which external or internal liquidators are rewarded when they close positions which are incurring excessive loss. In every case, the traders use their margin to refund the liquidity lost due to a losing trade. All trades are performed on a best-execution basis, searching through various external dex's. Some examples of supported dex's are Uniswap V2, Uniswap V3 and Sushiswap. Ithil does not have any internal dex.
\end{abstract}

\section{Introduction}
Margin trading is an ancient and famous financial tool which allows traders to invest more than their available liquidity, thanks to a system called \textit{leverage}. This makes it possible to profit from market movements even if the immediate funds availability is relatively low. On the other hand, this comes with the risk of losing more than in the case of a classical, non-leveraged investment.

In this workflow, the trader posts a \textit{margin} on the platform, which serves as a warranty against possible losses, and obtains the possibility of investing a part of the platform's liquidity which is typically higher than the margin he or she posted, under the payment of a fee. The ratio between the tradable amount and the margin is called the \textit{leverage}: a leverage of 10x means that the trader can invest 10 times the posted margin. The fees generated by such activity are then distributed between the \textit{liquidity providers (LPs)}, i.e. investors who deposited their tokens to make the margin trading activity possible.

The main difference between margin trading and an undercollateralized loan, is that the borrowed funds are never transferred to the trader: they are just locked in an investment within the platform, which must be closed in order for the trader to withdraw any gain.

\subsection{Comparison with other DeFi protocols}
A few protocols, offering the possibility to perform margin trading, are already available in mainnet. Ithil stands out among them due to many significant differences, between which there are the following.
\begin{enumerate}
\item The usage of a single vault to collect liquidity allows LPs to deposit any ERC20 token on the platform without paying any gas beyond the transaction: as long as there is a trade on such token, the fees generated will be distributed to the LP, generating passive income.
\item Fees are generated in the tokens used to trade, therefore LPs are rewarded with the same token they provided, instead of with the platform governance token. This reduces exchange risk related to the market value of the platform's governance token.
\item An innovative, three-layer liquidation system protects the investors from adverse market movements, and opens interesting arbitrage opportunities for liquidators, which in turn creates a positive feedback system: the more liquidators act, the healthier the vault is.
\item Very flexible investment system which allows to open, close and modify the position in many different ways, thus offering the best-in-market user experience for traders.
\item Unused liquidity is staked using a gas-efficient automatic staking algorithm: this increments dramatically the APY of the protocol.
\end{enumerate}

Although Ithil's contracts are not upgradable, some governance functions allow the community and the owners to change some of the state parameters (see further in this document), in order to calibrate them with the specific market conditions.

\section{Margin trading}

Margin trading is the core business of the platform, being the main source of income for Ithil's community and for LPs. Traders can use the platform's liquidity to perform any supported investment, and to gather high profits thanks to the leverage.

\subsection{Margin}
The \textit{margin} is a lump sum of tokens, which the trader deposits (\textit{posts}) into the contract in order to fund his or her trading activity. It is also the available tokens that can be withdrawn by the trader to realise the gains. The margin can be either \textit{free}, i.e. ready to be withdrawn or used for an investment, or \textit{locked} in a position (see further). The free margin does not change with time, while the locked one incurs market risk as it is involved in the trader's investment decision.
Traders can post margin in any token, also more than one at the same time. However, any order (see further) needs to have a unique token to be locked as margin in the forthcoming position.

We will consider the example of Theo, a trader, which posts $1000$ \verb|DAI| as margin. At the beginning the free margin is thus $1000$ \verb|DAI|, while no margin is locked in any position.

\subsection{Orders}\label{orderSubsection}
An \textit{order} is the implementation of an investment decision by the trader. It is a Solidity struct with the following members.
\begin{itemize}
\item \verb|address marginToken| the token posted as a margin.
\item \verb|address investmentToken| the token on which the trader wants to invest in.
\item \verb|uint256 margin| the amount of margin to be locked.
\item \verb|uint256 amount| the size of the order, i.e. the amount of tokens to buy or sell in order for the trade to be performed.
\item \verb|uint16 slippage| the maximum slippage the swap can experience without reverting.
\item \verb|bool buyExact| whether the order prescribes to buy an exact amount of tokens.
\item \verb|bool short| whether the order is for shorting the investment token.
\end{itemize}

For example, Theo might want to go long \verb|WETH|, thus betting on an \emph{evaluation} of the \verb|WETH| token with respect to \verb|DAI|. He then decides to buy $5000$ \verb|DAI| worth of \verb|WETH|, locking $500$ \verb|DAI| as a margin and accepting a $1$\% slippage. Theo is thus performing an investment with a 10x leverage, and if the order will not revert, his free margin will become $500$ \verb|DAI|, while the rest of his margin will be locked in the position that will be created after submitting the order.

TODO: explain internal swap system, dex search, maximum leverage, price impact, shorting.
 
\subsection{Positions}\label{positionSubsection}
A \textit{position} is the registration of one or several orders whose submission completes successfully. It is a Solidity struct with the following members.
\begin{itemize}
\item \verb|address trader| the position's owner.
\item \verb|address marginToken| the token locked as margin in the position.
\item \verb|address investmentToken| the token which the trader is investing on.
\item \verb|uint256 margin| the margin locked in the position.
\item \verb|uint256 entitlement| the amount of tokens obtained from the dex via the swap. This equals the maximum amount of tokens which can be moved by the trader after the position is opened.
\item \verb|uint256 debt| the amount of tokens given to the dex via the swap. This equals the amount of tokens which the trader owes to Ithil after the position is opened.
\item \verb|uint256 startTime| the moment the position has been opened.
\item \verb|uint256 fees| the fees produced so far by the position.
\item \verb|bool short| whether the position is long or short.
\end{itemize}
Getting back to our example, assume the market exchange rate at the moment Theo is performing the swap is $2500$ \verb|DAI| per \verb|WETH|. Assuming the trade incurs no slippage, the position will have an \verb|entitlement| of $2$ (\verb|WETH|) and a \verb|debt| of $5000$ (\verb|DAI|), while we recall from \ref{orderSubsection} that the \verb|margin| locked by the trader is $500$ (\verb|DAI|).

Let us assume that, after some time, the market price of \verb|WETH| has raised by $20$\% reaching $3000$ \verb|DAI| per \verb|WETH|. At this point, Theo may decide to \emph{close} the position he opened, exchanging his entitlement of $2$ \verb|WETH| to the dex, which will return (assuming no slippage) $6000$ \verb|DAI|. At this point, Theo must repay the position's \verb|debt|, thus getting $1000$\verb|DAI| of gain from this swap, assuming no time fees and no fixed fees (see \ref{feeSubsection}). Such gain is transferred to Theo's free margin, and the margin locked in the position is unlocked: after the position has been closed, Theo has $20000$ \verb|DAI| worth of free margin, which he can either withdraw or use for other investments.

Open positions can be modified at any time, either performing an additional swap in one direction or the other, allowing to increase or decrease the entitlement or the debt, or by locking some more margin in it, or again withdrawing part of the position margin, to adjust the trader's risk appetite towards a possible liquidation (see \ref{liquidationSubsection}). In all cases, a position entitlement and debt cannot exceed the maximum set up for the margin locked in. 

\subsection{Fees}\label{feeSubsection}

Fees in the event of a swap are always paid in the token borrowed to perform a swap; in particular, it changes if the order is long or short.
Fees are state variables of the protocol; in particular, they can be changed by Ithil's community. For each token, we have a \verb|Fees| struct composed by the following members.
\begin{itemize}
\item \verb|uint16 fixedFee| a fixed percentage fee for the tokens borrowed for a trade.
\item \verb|uint16 timeFee| a time-based fee which reflects the fact that the tokens spent to perform an investment are locked for the LPs (see \ref{LPSection}).
\item \verb|uint16 liquidationFee| the extra fee which is applied if a liquidation event occur (see \ref{liquidationSubsection}).
\end{itemize}

The fees are then stored in a private \verb|mapping(address => Fees) fees|, which associates to all tokens the relative fees. All members are percentages expressed in basis points, i.e. a value of $1$ corresponds to a fee of $0.01$\%. Moreover, the \verb|timeFee| is expressed on a \textit{daily basis}, with no compounding, and fixed fees are discounted 

In \ref{positionSubsection}, we saw how a swap of $5000$ \verb|DAI| was performed. Assuming a fixed fee of $0.2$\% for the \verb|DAI| token, and a time fee of $0.01$\%, and assuming the position has been kept for $10$ days, Theo must pay in total $0.3$\% of the swapped amount, i.e. $15$ \verb|DAI|, at the closure of the position. This is simply subtracted from Theo's free margin, which will be $1985$ \verb|DAI| when the position is closed (rather than $2000$ \verb|DAI|).

If a position is modified, the time fees generated so far go to populate the \verb|fees| member of the position, and the \verb|startTime| of the position is zeroed. In this way, only the debt effectively locked in a position generates time fees.

\subsection{Liquidation}\label{liquidationSubsection}

Since the amount of tokens swapped by Ithil is greater than the amount provided by the trader in the form of margin, the protocol would be exposed to an exchange risk if there were no liquidation system. This system will close the positions which are incurring a loss which is dangerously close to attacking the protocol's liquidity, thus protecting the LPs and at the same time putting a stop loss for the traders. Liquidation can therefore be thought of as a maintenance process to assure the health of the platform.

In line with the DeFi philosophy, Ithil outsorces liquidation to the public. Liquidating an open position means closing it, which in turn amounts to performing a swap in the opposite direction with respect to the one used to open that position.

\subsubsection{Risk factor}\label{riskFactorSubsubsection}

A position is \textit{liquidable} if the quoted value of its entitlement, plus the margin, minus de quoted value of its debt, all expressed as an amount of margin tokens, is lower than a certain fraction, called the \textit{risk factor}, of the position's margin. Therefore, assuming we are having a long position, calling $P$ the price of a unit of investment token expressed as amount of margin tokens, and $r$ the risk factor for the investment token, then the position is liquidable if
\begin{equation}\label{liquidableEq}
\text{entitlement} \times \text{price} + \text{margin} - \text{debt} - \text{fees} < r \times \text{margin}
\end{equation}
where "fees" are all the fees generated so far, except liquidation fees (see \ref{liquidationProcessSubsubsection}).
Risk factors are \textit{token-pair based}: the more volatile the token pair, the higher the risk factor. The main purpose of a high risk factor is to protect the protocol's liquidity from further adverse market movements occurring before the actual liquidation occurs, or during the closure of the position.

\subsubsection{The liquidation process}\label{liquidationProcessSubsubsection}

External liquidators can trigger the liquidation process after posting a margin on the platform corresponding to the tokens to be liquidated. A contract function loops through all the positions with a given margin and investment token, checking for the liquidability of each one and registering the positions' ID. The entitlement of all such liquidable positions (the debt for the short ones) is then summed, although it cannot exceed a maximum amount given by the liquidator and capped depending on the liquidator's margin (a typical amount is $100$ times the liquidator's margin, converted in investment tokens if the positions to be liquidated are long). This is done to avoid excessive price impact which can weaken the effectiveness of the liquidation process (see later on this section).

Once the loop is finished, a single swap of the entire entitlement calculated in the loop is performed, and all liquidable positions are closed; this avoids the high gas costs of performing many swaps to liquidate many positions. Because of Equation \eqref{liquidableEq}, if the price does not change abruptly during the liquidation process, each of such position will have a positive residual margin once closed. A percentage of such residual margin is then transfered to the liquidator's one: this is the \textit{liquidation fee} explained in \ref{feeSubsection}, while the rest of the margin is freed up and transferred to the trader's free margin.

\section{Liquidity providers}\label{LPSection}

The trading activity cannot happen without a large protocol's liquidity, therefore Liquidity providers (LPs) are of greatest importance for Ithil. Investors can provide liquidity and profit from the trading fees: the greatest majority of the fees is indeed redistributed to the LPs themselves, while a smaller amount is distributed to Ithil's community (precise amounts yet to be defined).

The fee redistribution process is conceived to respect a few core features:

\begin{itemize}
\item Absence of arbitrage opportunities.
\item Gas efficiency.
\item Competitive APY.
\item Very low risk of capital loss.
\item High fungibility of the LP position.
\end{itemize}

In order to achieve this, Ithil mints fungible wrapped tokens which serve as a proxy of the deposited liquidity, and of the claiming rights of the particular token deposited. \textit{These tokens have nothing to do with Ithil's governance token}. In particular, each liquidity provider is always rewarded \textit{in the same token he or she has deposited}.

\subsection{Wrapped tokens}\label{wrappedTokenSubsection}

When a LP provides some amount of a particular token, a certain amount of wrapped tokens is minted by Ithil (for the precise amount, see section \ref{feeDistributionSubsection}) to reflect the amount deposited and the type of token deposited. Such tokens are then transfered to the LP's wallet and can be exchanged freely.

Such wrapped tokens can then be redeemed by making Ithil burn them and transfer (if immediately available: see \ref{automatedStakingSubsection}) a corresponding amount of original tokens to the wrapped tokens' owner. Such amount is calculated in Section \ref{feeDistributionSubsection}, and it corresponds to the original deposited amount, plus a portion of the fees depending on the wrapped tokens burnt and the total supply. In this way, only fees generated \textit{after} the minting can be taken by the LP who triggered the minting, thus allowing no arbitrage opportunity and a fair fee redistribution.

\subsection{Total wealth}\label{totalWealthSubsection}

By \textit{total wealth} for a particular token, we mean the amount of that token which has been deposited or generated as fees or after staking. In particular, the liquidity posted as margin or generated as a result of a swap is not considered. Moreover, virtual liquidity coming from staking shares (see \ref{automatedStakingSubsection}) is considered although not immediately available.

\subsection{Fee distribution}\label{feeDistributionSubsection}

The fee distirbution is directly related to the amount of tokens minted and burned during the deposit and withdrawal process. At the moment of the deposit, Ithil will mint an amount of wrapped tokens calculated as follows:

\begin{equation}\label{mintedAmount}
\text{minted amount} = \frac{\text{deposit} \times (\text{total w-token supply})}{\text{total wealth}}
\end{equation}
where the total wealth and the total wrapped token supply are as of \textit{before} the deposit. In the case the total wealth is $0$ (i.e. no investor has deposited that particular token yet), then we adopt a $1:1$ rate of minted token to deposited token: 
\begin{equation}\label{mintedAmountZeroWealth}
\text{minted amount} = \text{deposit,} \ \ \ \text{ if total wealth } = 0.
\end{equation}

Similarly, when the wrapped tokens are burned through Ithil, an amount of original tokens is transferred to the burner, following this equation:

\begin{equation}\label{transferEq}
\text{transferred amount} = \frac{ \text{total wealth}}{\text{total w-token supply}} \times\text{burned amount}
\end{equation}
where the total wealth and total w-token supply are as of \textit{before} the transfer.

\subsection{Automated staking}\label{automatedStakingSubsection}

The main source of income for the investors are the trader's fees, coming from the trading activity on the particular token posted. For better risk management, and at the same time to obtain a return on the idle liquidity, Ithil implements a staking algorithm using Yearn Finance as vault.

\subsubsection{Thresholds}

Continuously staking and unstaking of the necessary liquidity would cause a high gas cost for the traders and investors, thus resulting in a high inefficiency of the protocol. Instead, we stake the total wealth following big \textit{thresholds} at an interval of $20$\% each. More precisely, the following rules allow for the construction of a staking algorithm.
\begin{itemize}
\item At least $20$\% of the liquidity needs to be immediately available for the investors to withdraw (with the burning procedure described in \ref{feeDistributionSubsection}).
\item At least $20$\%, and at most $80$\%, of the liquidity must be available for the trading activity. Respecting this, such available liquidity must be \textit{minimized}.
\item At most $60$\% of the liquidity will be staked.
\item When staking or unstaking, exactly $20$\% of the total liquidity must be moved.
\end{itemize}

Therefore, each time an event modifying the liquidity or the trading needs occurs (posting / withdrawing the margin, deposit / withdraw of liquidity by the LPs...), a check is done, and if one of the previous criteria is not observed, $20$\% of the total liquidity will be staked or unstaked based on the particular condition.

\end{document}